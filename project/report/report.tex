\documentclass[a4paper]{article}

\usepackage{tecnico_relatorio}

\usepackage{textcomp}
\usepackage[hypcap]{caption} % makes \ref point to top of figures and tables
%\usepackage{rotating}
\usepackage{float}
\usepackage[nottoc]{tocbibind}
\usepackage[utf8]{inputenc}
\usepackage{graphicx}
\usepackage[justification=centering]{caption}
\usepackage{listings}
\usepackage{indentfirst} % indent first paragraph in section
\usepackage{geometry}	% margins

\begin{document}
	\newgeometry{left=4.5cm,right=4.5cm}

	\trSetImage{img/tecnico_logo}{6cm} % Logotipo do Técnico
    
    \trSetCourse{Mestrado em Engenharia Electrotécnica \\e de Computadores}
    
	\trSetSubject{Programação de Sistemas}
	
	%\trSetType{2ª Parte}

	\trSetTitle{Projecto}

	\trSetBoxStyle{0.3}

	\trSetGroupNo{Grupo 3}

	\trSetAuthorNr{2}

	\trSetAuthors
	{
		\begin{center}
			Gonçalo Ribeiro

			73294
		\end{center}
	}{
		\begin{center}
			Miraje Gentilal

			73547
		\end{center}
	}

	\trSetProfessor{Prof. João Nuno de Oliveira e Silva}

\trMakeCover
    
	\restoregeometry

	\tableofcontents
	\pagenumbering{gobble}

	\pagebreak

	\pagenumbering{arabic}
	\setcounter{page}{1}


	\section{Server Architecture}

	\subsection{Components Implementation}

	\subsection{Components Communication}



	\section{Parallelism implementations}



	\section{Server data structures}

	\subsection{Race conditions}



	\section{Fault tolerance handling}



	\section{Communication}



	\section{Unitary testing}



	\section{Conclusions and comments}



\end{document}
